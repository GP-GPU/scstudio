% $Id: $

\documentclass{article}
\usepackage[plainpages=false, pdfpagelabels]{hyperref}
\usepackage[pdftex]{graphicx,color}
\usepackage{algorithmic}
\usepackage{algorithm, verbatim}
\newtheorem{defn}{Definition}
\newtheorem{thm}{Theorem}
\newtheorem{lem}{Lemma}
\newcommand{\type}{{\tau}}
\newcommand{\proc}{{P}}
\newcommand{\procSet}{{\cal P}}
\newcommand{\events}{{\cal E}}
\newcommand{\match}{{\cal M}}
\newcommand{\coregs}{{\cal C}}
\newcommand{\genord}{{\cal G}}
\newcommand{\MSCmap}{{\it L}}
\newcommand{\MSCs}{{\cal L}}
\newcommand{\timerSet}{{\cal T}}
\newcommand{\timerMSGSet}{{{\cal T}_{MSG}}}
\newcommand{\intervalSet}{{\cal I}}
\newcommand{\intConstraintSet}{{\cal K}}
\newcommand{\intConstraintMSGSet}{{\cal K}_{MSG}}
\newcommand{\maxEvents }{{\it{Max}}}
\newcommand{\minEvents }{{\it{Min}}}
\newcommand{\solutionSet }{{\cal S}}
\newcommand{\timingAssignment }{{\iota}}
\newcommand{\timingAssignmentMSG}{{{\iota}_{MSG}}}
\usepackage{amsmath, amsthm, amssymb}

\begin{document}

\section*{Beautify}
The goal is to redraw a given \emph{bMSC} $ M =
(E,<,\procSet,\type,\proc,\match,\coregs,\genord,\intConstraintSet, \timerSet)$
such that all messages will be horizontal if possible.

%\begin{algorithm}
%\caption{Beautify(bMSC $M$)}
\begin{algorithmic}[1]

\STATE
\STATE \textbf{Input} bMSC $M$
\STATE \textbf{Output} bMSC $M$ with horizontal messages (if possible)
\STATE 

\STATE set $horizontal \leftarrow \{\}$;
\STATE set $oblique \leftarrow \{\}$;
\STATE event $e, e', e\_global$;
\STATE bMSC $N$;
\STATE int $gap$;
\STATE instance $p, i$;
\STATE

\STATE $M.put\_events\_up()$;
\FORALL{$q$ $\in$ instances of $M$}
\STATE \COMMENT {events those messages have not been worked yet}
\FOR{$e$ $\in$ events of $q$, $e.get\_message() \not \in horizontal$,  $e.get\_message() \not \in oblique$ }
\STATE $e' \leftarrow e.mached\_event()$;
\STATE $p \leftarrow e'.get\_instance()$; 
\STATE
\STATE \COMMENT {horizontal messages}
\IF{$e.get\_y() == e'.get\_y()$}
\STATE $horizontal.add(e.get\_message())$; 
\ENDIF 
\STATE

\STATE \COMMENT {nonhorizonte message when message is going up}
\pagebreak
\IF{$e.get\_y() > e'.get\_y()$}
\IF{none event on $p$ between $e.get\_y()$ and $e'.get\_y()$}
\STATE $e'.set\_y(e.get\_y())$; 
\STATE $horizontal.add(e.get\_message())$;
\ELSE
\STATE $global\_e \leftarrow e$; 
\STATE $N \leftarrow M.duplication()$;
\STATE $gap \leftarrow (e.get\_y() - e'.get\_y())$;


\FORALL{$i$ $\in$ instance of $M$}
\STATE $i.set\_atribut(0)$;
\ENDFOR

\IF{$M.shift(p, e', e, gap, global\_e) == false$}
\STATE $M \leftarrow N$;
\STATE $oblique.add(e.get\_message())$;
\ELSE 
\STATE $e'.set\_y(e.get\_y())$;
\STATE $horizontal.add(e.get\_message())$;
\ENDIF

\ENDIF
\ENDIF
\STATE

\STATE \COMMENT {nonhorizonte message when message is going down}
\IF{$e.get\_y() < e'.get\_y()$}
\IF{none event on $q$ between $e.get\_y()$ and $e'.get\_y()$}
\STATE $e.set\_y(e'.get\_y())$; 
\STATE $horizontal.add(e.get\_message())$;
\ELSE
\STATE $global\_e \leftarrow e'$; 
\STATE $N \leftarrow M.duplication()$;
\STATE $gap \leftarrow e'.get\_y() - e.get\_y()$;


\FORALL{$i$ $\in$ instance of $M$}
\STATE $i.set\_atribut(0)$;
\ENDFOR

\IF{$M.shift(q, e, e', gap, global\_e) == false$}
\STATE $M \leftarrow N$;
\STATE $oblique.add(e.get\_message())$;
\ELSE 
\STATE $e.set\_y(e'.get\_y())$;
\STATE $horizontal.add(e.get\_message())$;
\ENDIF

\ENDIF
\ENDIF
\STATE

\ENDFOR
\ENDFOR

\end{algorithmic}
%\end{algorithm}

\subsection*{Description of algorithm Beautify(bMSC $M$)}
At first we declare some helping variables.\\
	$horizontal$		 set of straight arrows \\
	$oblique$ set of oblique arrows \\
	$e$			left event of examined arrow \\
	$e'$		right event (second end) of examined arrow \\
	$p$			instance where event e' is lying \\
\\
Then we call function $put\_events\_up()$ on bMSC $M$. Now we have all events up and for straighten the arrows we will move them down. \\
For all events of all instances we solve three different situations: \\
\begin{itemize}
\item The first situation is when the arrow is straight (lines 19 - 22). It is when $y$ coordinates of both events of arrow are equal. Then we only add this arrow to the variable $horizontal$. \\
\item The second situation is when the arrow is going up (lines 24 - 44). It is when $y$ coordinate of right event ($e'$) is lower than $y$ coordinate of left event ($e$). If there is no event on instance $p$ (instance where $e'$ is lying) between $y$ coordinates of both events then we can move $e'$ without any other changes and add arrow into $horizontal$. If there is any events we have to move them and all others which relate with $e'$. For moving all events we use function $shift$ and we will do it on bMSC $M$. If it is impossible to move events (there is any crossing) the function $shift$ returns $FALSE$. Then we put $N$ (copy of bMSC $M$ withoud changes) into $M$. When there is not crossing, we move event $e'$ and add arrow to set $horizontal$. Besides this on lines 30 and 32 - 35 we declare helping variable $global\_e$ and set atributs of instances. It is needed for function $shift()$.
\item The last situation is when the arrow is going down (lines 46 - 66). It is when $y$ coordinate of right event ($e'$) is bigger than left event ($e$). If there is no event on instance $q$ (instance where $e$ is lying) between $y$ coordinates of both events then we can move $e$ and add arrow into $horizontal$. Else we copy $M$ into helping variable $N$ and move all events with function $shift$. If it is impossible (crossing), we put $N$ into $M$. If it is possible, we will move $e$ into same level as $e'$ and add arrow into $horizontal$.  
\end{itemize}
   



\begin{algorithm}
\caption{shift(instances $q$, event $e'$, event $e$, int $gap$, event $global\_e$)}
\begin{algorithmic}[1]

\STATE \textbf{Input} bMSC whose events on $q$ are not moved
\STATE \textbf{Output} TRUE - if it is possible to move events (no crossing), FALSE - if it is impossible to move events (crossing)
\STATE

\STATE event $f, f'$;
\STATE instance $j$;
\STATE

\STATE \COMMENT {events on q, under e'}
\FOR{(events $f$; $f$ $\in$ $q$; $f.get\_y > e'.get\_y$)}
\STATE $f' \leftarrow f\_mached\_event()$;
\STATE $j \leftarrow f'.get\_instance()$; 
\STATE \COMMENT {oblique arrow can be obliquer}
\IF{$f.get\_message()$ $\not \in horizontal$}
\STATE $f.set\_y(f.get\_y() + gap)$;
\ELSE

\FOR{(events $h; h \in j; h.get\_y() > f'.get\_y()$)}
\IF{$h == global\_e$}
\STATE return false;
\ENDIF
\ENDFOR
\STATE

\IF{$j.get\_atribut() == 0$}
\STATE $j.set\_atribut(1)$;
\IF{$shift(j,f',f,gap,global\_e)==$false}
\STATE return false;
\ENDIF
\ENDIF

\STATE $f.set\_y(f.get\_y() + gap)$;
\STATE $f'.set\_y(f'.get\_y() + gap)$;

\ENDIF

\ENDFOR

\STATE return true;

\end{algorithmic}
\end{algorithm}

\subsection*{Description of algorithm shift(instances $q$, event $e'$, event $e$, int $gap$, event $global\_e$)}
We get some atributes.\\
$q$ is instance on which there is the event which we want to move \\
$e'$ is the event which we want to move \\
$e$ is mached event of $e'$ \\
$gap$ is difference between $y$ coordinations of both events \\
$global\_e$ is event which will help us to detect crossing two arrows. Its value does not change. \\
\\
Function $shift$ tries to move events which are below $e'$ (marked as $f$) and if it is necesarry then also tries to move events relate with $f$ (marked as $f'$).\\
It does not matter how much oblique a nonhorizontal arrow is. So if arrow with terminal events $f$ and $f'$ does not belong to $horizontal$ we can move event $f$ with no other changes (we do not move $f'$). Then the function retuns TRUE.\\
If arrow with terminal events $f$ and $f'$ belong to $horizontal$ we have to find out if there is any crossing between any arrows (it is impossible to straighten arrow) or not. If there is any crossing, one of events below $f'$ will be global\_e. We search all events below with recursion of function $shift$.  \\
Because we do not want to search one instance twice, we will use $atribut$ of instances. If $atribut$ is equal to 1 then we have already searched it.\\
If we find any crossing the function will return FALSE. 


\begin{algorithm}
\caption{put\_events\_up()}
\begin{algorithmic}[1]

\STATE int $step \leftarrow 10$;
\STATE \COMMENT {distance between events}
\FORALL{$q$ $\in$ instances of $M$}
\FORALL{$e$ $\in$ events of $q$}
\STATE $e.set\_y(step)$;
\STATE $step \leftarrow step + 10$;
\ENDFOR
\STATE $step \leftarrow 10$;
\ENDFOR



\end{algorithmic}
\end{algorithm}

\subsection*{Description of algorithm put\_events\_up()}
For all instances we put their events up. Between two events of one instance there will be a constant gap. We are doing this because we want to move events only in one direction (down).


\end{document}
